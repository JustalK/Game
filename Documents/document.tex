\documentclass{article}

\usepackage{color}
\usepackage{graphicx}
\usepackage{tabularx}
\usepackage[frenchb]{babel}
\usepackage[utf8]{inputenc}
\usepackage[T1]{fontenc}
\usepackage{lmodern}

\usepackage{geometry,wrapfig,lipsum}
 \geometry{
 top=20mm,
 bottom=20mm,
 }


\title{Evaluation}
\author{Justal Kevin}
\date{30/10/2015}
\renewcommand{\contentsname}{Table des mati\`eres} 
 
\newcommand\invisiblesection[1]{%
  \refstepcounter{section}%
  \addcontentsline{toc}{section}{\protect\numberline{\thesection}#1}%
  \sectionmark{#1}} 
 
\begin{document}

\begin{center}
\textbf{\Huge{Jeux}}\\
\line(1,0){300}\\
En route pour le million !\\
\vspace{3cm}
\includegraphics[width=\textwidth]{1}\\
\vspace{3cm}
\textbf{JUSTAL KEVIN}\\
2015-2016\\
\vspace{3cm}
\textbf{\color{blue}{\underline{justal.kevin@gmail.com}}}\\
\end{center}

\newpage
\section{Requis}
Le jeu doit premièrement être \textbf{simple}. C'est à dire qu'il faut être capable de parcourir le jeu et de pouvoir jouer avec un simple doigt. Il faut qu'il soit simple a jouer mais que le jeu demande un certains niveaux pour obtenir un gros score\\

Le jeu doit avoir une \textbf{difficulté croissante exponentielle}, c'est à dire qu'il faut que la difficulté du jeu doit être relativement simple dans les premiers niveaux puis progressivement la difficulté doit monter de manière violente.\\

Le jeu doit être basé sur \textbf{le rythme et le timing}. Il s'agit là du système de jeu sur mobile qui a le mieux fonctionné dans le monde. Il faut que la musique du jeu colle avec les appuies sur le clavier.\\
\section{Technologies}
\subsection{Phonegap}
Afin de faire une application avec les technologies que je maitrise le mieux et aussi afin d'etre sur toutes les plateformes.
\subsection{JQuery Mobile}
Afin de gerer les évenements du joueur et d'etre cross-plateforme.
\section{Analyse de l'existant}
\subsection{Flappy bird}
People in the same classroom can play and compete easily because [Flappy Bird] is simple to learn, but you need skill to get a high score

\subsection{Super Graviton}
\subsection{The impossible game}
\subsection{Geometry Dash}
\subsubsection{Musiques}
Chaque niveaux a un son unique. Les sons sont généralement de 3 à 4 minutes alors que les niveaux durent 1.5 à 2 minutes.

\section{Idées}
Un jeu ou on defille vers la gauche et on ajoute des obstacles au fur et a mesure du temps. Pour les deplacement, on appuie sur l'ecran pour aller en haut ou en bas. Si on appuie pendant que l'on a pas touche le bord oppose au premier deplacement, on retourne vers celui ou on etait avant.\\


Le jeu doit joué avec les couleurs lorsque le cube arrive sur une plateforme.\\

Un simple cube est suffisant pour le graphisme, avec une tete dedans ;)\\

OPTIONNEL : Le jeu doit pouvoir ajouter un son de l'utilisateur et créer un niveau dessus.\\

\end{document}